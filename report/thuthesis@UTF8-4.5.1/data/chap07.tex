\chapter{结论}\label{chapter:conclusion}
本论文介绍了分布式二级哈希表,详细阐述了其架构设计和实现细节。分布式二级哈希表
采用分布式领域一些经典技术,在分布式系统上实现了二级哈希的存储语义,为邮件系
统、个人云存储等上层应用提供后端的存储支持。分布式二级哈希表由配置服务器、客户
端和存储服务器集群构成。其中,配置服务器监控各存储结点的运行状况,集中管理控制
信息,并通过一致性哈希算法实现数据的分配的备份。客户端采用分层结构,利用远程过
程调用从配置服务器获取数据的目标存储服务器的IP地址,借助线程池实现请求的并行异
步执行,并通过$R+W>N$语义保证了数据在系统中的弱一致性。单个存储结点上采用开源
项目Redis,实现了本地二级哈希存储,具备很高的性能。

分布式二级哈希表在设计之初,就本着尽量提高系统运行效率的根本原则。实验证明,当
前的系统实现也确实达到了较高的性能。当然,分布式二级哈希表仍然存在性能提升的空
间,今后的工作可以从增强系统的可扩展性和容错性入手,进一步完善和改进分布式二级
哈希表。
