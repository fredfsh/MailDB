\ctitle{分布式二级哈希表}
% 根据自己的情况选,不用这样复杂
\cdegree{工学硕士}
\cdepartment[计算机]{计算机科学与技术系}
\cmajor{计算机科学与技术}
\cauthor{冯时} 
\csupervisor{陈文光教授}
% 如果没有副指导老师或者联合指导老师,把下面两行相应的删除即可。
\cassosupervisor{陈康副教授}
% 日期自动生成,如果你要自己写就改这个cdate
%\cdate{\CJKdigits{\the\year}年\CJKnumber{\the\month}月}

% 定义中英文摘要和关键字
\begin{cabstract}
随着互联网的发展和云存储的兴起,传统的分布式哈希表越来越难以满足丰富的个人在线
存储应用对存储后端的需求。二级哈希表存储的是具有二级索引的数据,因此能够提供更
丰富的语义。本论文阐述了分布式二级哈希表的架构设计和实现细节,说明了性能对于分
布式存储系统的重要性。分布式二级哈希表综合了分布式系统领域的一些经典技术,利用
配置服务器集中管理控制信息,并采用一致性哈希进行数据的分配和备份。
\end{cabstract}

\ckeywords{分布式, 哈希表, 存储, 性能, 一致性哈希}

\begin{eabstract} 
With the development of Internet and boom of cloud storage, traditional
distributed hash table is too simple to support the upper layer applications
which provide personal online storage service. A two-level DHT stores data
with a major index plus a secondary one, resulting in more complex semantics.
This thesis describes both architecture design and implementation details of
two-level DHTs, and comes to a conclusion that high performance and efficiency
is fundamental aspect of a distributed system. The two-level DHT is a
combination of classical technologies in distribute computing, with a config
server to monitor the cluster. It also employs consistent hashing to solve
data partition and replication.
\end{eabstract}

\ekeywords{distributed, hash table, storage, performance, consistent hashing}
