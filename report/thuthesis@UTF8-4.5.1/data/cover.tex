\ctitle{分布式二级哈希表}
% 根据自己的情况选,不用这样复杂
\cdegree{工学硕士}
\cdepartment[计算机]{计算机科学与技术系}
\cmajor{计算机科学与技术}
\cauthor{冯时} 
\csupervisor{陈文光教授}
% 如果没有副指导老师或者联合指导老师,把下面两行相应的删除即可。
%\cassosupervisor{陈康副教授}
% 日期自动生成,如果你要自己写就改这个cdate
%\cdate{\CJKdigits{\the\year}年\CJKnumber{\the\month}月}

% 定义中英文摘要和关键字
\begin{cabstract}
随着互联网的发展和云计算的兴起,邮件系统、个人在线相册等云存储应用对底层的存储
后端提出了越来越高的要求。系统一方面能够存储和管理大容量的数据,具备较高的扩展
性;另一方面又要保证在对海量数据的操作能够迅速得到响应,运行效率较高。传统的分
布式存储系统由于支持的语义比较简单,或者系统架构复杂,不能很好的适应云存储应用
的需求。分布式二级哈希表存储具有二级索引的数据,其语义能够满足上层应用的基本需
求,同时整个系统结构简单,运行高效。本论文的主要贡献如下:
\begin{enumerate}
  \item 详细阐述了如何通过综合分布式领域的经典技术,设计并实现了分布式二级哈希
  表存储系统。其中,数据在集群中的分布和备份采用一致性哈希算法,数据的本地存储
  依赖开源项目Redis,操作的异步化用到了线程池。
  \item 说明了运行效率是分布式系统的重要设计原则之一,并展示了在分布式二级哈希
  表中,如何通过对系统作出一系列优化来提升系统性能。
\end{enumerate}
\end{cabstract}

\ckeywords{分布式, 哈希表, 存储, 性能, 一致性哈希}

\begin{eabstract} 
With the development of Internet and boom of cloud computing, cloud storage
applications, say mail system and personal online photo albums, rely more and
more heavily on the underlying backend storages. The system should not only
store and manager huge amount of data, with high scalability, but also limit
operation time, with high functional efficiency. Traditional distributed
storage systems either provide too simple semantics, or suffer from high
overheads of operations with relatively complicated system designs, hence they
are more and more unsuitable for the cloud storage applications. A two-level
distributed hash table stores data with a major index plus a secondary one.
This data structure along with the operations upon them support well enough
for the upper layer applications. Besides, the system benefits from a simple
and clear architecture design and functions effectively. The present thesis
makes the following major contributions:
\begin{enumerate}
  \item It describes in details how the two-level DHT is designed and
  implemented combining classic technologies in the area of distribute system.
  Data are partitioned and replicated subject to consistent hashing algorithm.
  Local storage is handled by Redis, an open source project. Operations are
  executed asynchronously with the help of a thread pool.
  \item It also shows that efficiency is one of critical principles in system
  design and a major concern during implementation and optimization of a
  distributed system.
\end{enumerate}
\end{eabstract}

\ekeywords{distributed, hash table, storage, performance, consistent hashing}
